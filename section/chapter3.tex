\chapter{Faizal Mutakin/1144078}
\begin{table}[h]
\caption{Tabel Kegiatan Harian IRC}
\centering
\begin{tabular}{|c|c|c|}
\hline
No&Tanggal&Kegiatan\\
\hline
1.&\multirow{2}{*}{18-02-2019}&1.persentasii adik tingkat\\
&&2.meriksa revisi laporan adik tingkat\\
&&3.pull request gis2019\\
\hline
2.&\multirow{2}{*}{25-02-2019}&1. Membuat repositori Modul Praktikum kelas Pemrograman III.\\
&&2. Menginstal python, pip, anaconda, dan belajar UUDMY.\\
&&3.pull request gis2019\\
\hline
3.&\multirow{2}{*}{26-02-2019}&1.Masuk kelas kecerdasan buatan\\
&&2.mebuat repositori AI\\
&&3.memberi nilai hasir pull requesh\\
&&4.pull request gis2019\\
\hline
4.&\multirow{2}{*}{27-02-2019}&1.Cek tugas pull reques 3c\\
&&2.Memberi nilai kecerdasan buatan\\
&&3.Mengoreksi adik tingkat yang tidak pull reques\\
&&4.pull request di buku informatika gis2019\\
\hline
5.&\multirow{2}{*}{27-02-2019}&1.evaluasi tugas mingguan di prodi\\
&&2.Menilai hasil UAS kelas 3A\\
&&3.pembagian tempat kebersihan di ruangan irc\\
&&4.menata ulang tempat komputer\\
&&5.input nilai kecerdasan buatan andri fajar sunandhar(1164065)\\
&& Imron Sumadireja(1164076) Yusniar Syarif Sidiq \\
&&6.pull request gis2019\\
\hline
4.&\multirow{2}{*}{27-02-2019}&1.Cek tugas pull reques 3c\\
&&2.Memberi nilai kecerdasan buatan\\
&&3.Mengoreksi adik tingkat yang tidak pull reques\\
\hline
\end{tabular}
\label{table:contoh}
\end{table}